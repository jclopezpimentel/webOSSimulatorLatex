%%
%% Copyright 2007, 2008, 2009 Elsevier Ltd
%%
%% This file is part of the 'Elsarticle Bundle'.
%% ---------------------------------------------
%%
%% It may be distributed under the conditions of the LaTeX Project Public
%% License, either version 1.2 of this license or (at your option) any
%% later version.  The latest version of this license is in
%%    http://www.latex-project.org/lppl.txt
%% and version 1.2 or later is part of all distributions of LaTeX
%% version 1999/12/01 or later.
%%
%% The list of all files belonging to the 'Elsarticle Bundle' is
%% given in the file `manifest.txt'.
%%

%% Template article for Elsevier's document class `elsarticle'
%% with numbered style bibliographic references
%% SP 2008/03/01
%%
%%
%%
%% $Id: elsarticle-template-num.tex 4 2009-10-24 08:22:58Z rishi $
%%
%%
\documentclass[preprint,12pt,3p]{elsarticle}
%\usepackage{graphicx, soul} 
\usepackage{hyperref}
\usepackage{multirow}
\usepackage{colortbl} % Biblioteca para el color
\usepackage{color} % Biblioteca para el color
\usepackage{xcolor} % Biblioteca para el color
%% Use the option review to obtain double line spacing
%% \documentclass[preprint,review,12pt]{elsarticle}

%% Use the options 1p,twocolumn; 3p; 3p,twocolumn; 5p; or 5p,twocolumn
%% for a journal layout:
%% \documentclass[final,1p,times]{elsarticle}
%% \documentclass[final,1p,times,twocolumn]{elsarticle}
%% \documentclass[final,3p,times]{elsarticle}
%% \documentclass[final,3p,times,twocolumn]{elsarticle}
%% \documentclass[final,5p,times]{elsarticle}
%% \documentclass[final,5p,times,twocolumn]{elsarticle}

%% if you use PostScript figures in your article
%% use the graphics package for simple commands
%% \usepackage{graphics}
%% or use the graphicx package for more complicated commands
%% \usepackage{graphicx}
%% or use the epsfig package if you prefer to use the old commands
%% \usepackage{epsfig}

%% The amssymb package provides various useful mathematical symbols
\usepackage{amssymb}
%% The amsthm package provides extended theorem environments
%% \usepackage{amsthm}

%% The lineno packages adds line numbers. Start line numbering with
%% \begin{linenumbers}, end it with \end{linenumbers}. Or switch it on
%% for the whole article with \linenumbers after \end{frontmatter}.
%% \usepackage{lineno}

%% natbib.sty is loaded by default. However, natbib options can be
%% provided with \biboptions{...} command. Following options are
%% valid:

%%   round  -  round parentheses are used (default)
%%   square -  square brackets are used   [option]
%%   curly  -  curly braces are used      {option}
%%   angle  -  angle brackets are used    <option>
%%   semicolon  -  multiple citations separated by semi-colon
%%   colon  - same as semicolon, an earlier confusion
%%   comma  -  separated by comma
%%   numbers-  selects numerical citations
%%   super  -  numerical citations as superscripts
%%   sort   -  sorts multiple citations according to order in ref. list
%%   sort&compress   -  like sort, but also compresses numerical citations
%%   compress - compresses without sorting
%%
%% \biboptions{comma,round}

% \biboptions{}
\newcommand{\Caro}[1]{\textcolor{magenta}{\textit{#1}}}

\journal{Computers \& Education}

\begin{document}

\tableofcontents
\newcommand{\pathpractices}{https://github.com/jclopezpimentel/webOSSimulatorLatex/activities/}
\newcommand{\pathexams}{https://github.com/jclopezpimentel/webOSSimulatorLatex/exams/}

\begin{frontmatter}
        %\title{Development an Operating System Simulator as an Strategy for Teaching Web Programming\tnoteref{label0}}
\title{Operating System Simulator Development as a Strategy for Teaching Web Programming\tnoteref{label0}}
\tnotetext[label0]{This is only an example}


\author[label1,label2]{Juan Carlos L\'opez Pimentel\corref{cor1}\fnref{label3}}
\address[label1]{Address One}
\address[label2]{Address Two\fnref{label4}}

\cortext[cor1]{I am corresponding author}
\fntext[label3]{I also want to inform about\ldots}
\fntext[label4]{Small city}

\ead{author.one@mail.com}
\ead[url]{author-one-homepage.com}

\author[label5]{Author Two}
\address[label5]{Some University}
\ead{author.two@mail.com}

\author[label1,label5]{Author Three}
\ead{author.three@mail.com}

        \begin{abstract}
The web is transformed quickly and also the technologies used for its development. 
This demands and provokes a challenge for teachers to reform traditional methods 
in the development of web applications. 
This article presents our experience and knowledge precisely in this type of courses.

The objective of this article is to describe a teaching method that is based on the 
personalized development of an operating system simulator on the web. 
In addition, some activities that students must complete to pass the course are shown. 
This article can also be used by teachers who are looking for a teaching strategy in 
Web Technologies courses within a curricular plan.
\end{abstract}

\begin{keyword}
%% keywords here, in the form: keyword \sep keyword
Programming, Web, Client-Server, Teaching and Learning Strategies.
%% MSC codes here, in the form: \MSC code \sep code
%% or \MSC[2008] code \sep code (2000 is the default)
\end{keyword}

\end{frontmatter}

%%
%% Start line numbering here if you want
%%
% \linenumbers

        \section{Introduction}
\label{sec:intro}

Currently the Web facilitates our life in several aspects. Applications such as 
email, chat, social networks, among others, allow people to be more connected 
and interrelated than in previous times; Internet banking, electronic commerce 
are examples of applications that save us time and effort. In terms of getting 
information we have also been favored, which a while ago it took us a long time 
to know, nowadays with the search engines allow us to have the information in a 
short time.

In education, the Web provides a wide range of resources and services focused on 
the teaching-learning process, but it also causes several challenges. For example, 
validate that the information is reliable. Many solutions of activities or class 
projects can be found on the Web, this being a real challenge for teachers, since 
the extra-class activities left by teachers could be being by the students, then 
the teacher would not have the certainty that the students are achieving the goal.

So, this era with the Internet demands and challenges teachers to reform their 
traditional courses and even more if the subject is related to programming or 
development of Web applications. The Web is reinvented every day, what was once 
considered technically impossible to do, nowadays it is easier, since with the 
passage of time the technologies used for their development have been innovated. 
Thus, teaching Web applications is a changing topic that can not remain static.

Some researchers emphasize much about the main challenges in web programming 
\cite{9Wang2009,7XinogalosK2012,4Liu2011}. Other authors have proposed different 
strategies to deal with it; some propose technologies that should be taught in a 
single course, \cite{4Liu2011,10Rosenbloom2017}, but with prior knowledge. Others 
propose dividing it in two \cite{9Wang2009}. In our case, we have considered some 
of such proposals, which will explained below, but focused on a single course.

We are inspired by the following hypothesis: with the customized development of a 
web-based operating system simulator, we can find the formula to generate dynamic 
solutions and thus achieve the necessary skills that students must acquire in Web 
Programming in such demanding times.

Our proposal is suitable as long as the students are trained in topics such as 
structured programming, object-oriented programming, data structure, and database 
notions. If the students already have some knowledge in web technologies it is 
desirable but it is not necessary.

The general objective of this article is to provide facilitators, and even students, 
a teaching-learning methodology for subjects related to Web Programming through 
strategies that combine the knowledge and experience of the teacher with an 
integrating project (simulator of an operating system) that students generate in a 
personalized way from the beginning to the end of the subject. In addition, students 
must submit activities, which we call evidences that are attached to the course 
syllabus and the integrating project, that students must complete to pass. All this 
will eventually lead to better skills in the students.

Although this proposal has been applied in different scenarios, the proposal 
presented here has been adjusted to 48 contact hours and 48 non-contact hours. 
To combat the restriction of time considering the amount of web technologies that 
currently exist and give students the main motivation to learn the State of the Art 
in this topic, we have carefully selected a set of important and representative topics 
that eventually lead to a set of tools currently used by developers.

%Sample text. Sample text. Sample text. Sample text. Sample text. Sample text. 
%Sample text. Sample text. Sample text. Sample text. Sample text. Sample text. 
%Sample text. Sample text. Citation of Einstein paper~\cite{Einstein}.

% \subsection{Sample subsection}
% \label{subsec1}

% Sample text. Sample text. Sample text. Sample text. Sample text. Sample text. 
% Sample text. Sample text. Sample text. Sample text. Sample text. Sample text. 
% Sample text. 

        %\section{Web Operating System Simulator as an Evidence in the Teaching-Learning Process}
\section{Operating System Simulator with Web Technology}
\label{sec:GralExplanation}

\subsection{Learning Units}
\label{ssec:learningUnits}
The subject comprises 16 weeks, each week involves 3 face-to-face hours and 
3 extra-class hours. Our restriction is to provide in a single course web 
programming topics of Front-end and Back-end, hence we have conformed the 
following units: 
i) Design, development and programming of web pages on the client's side; 
ii) Internet, WWW and HTTP; and 
iii) Programming of server-side web pages.

The first unit comprises three learning outcomes: 
i) Design web pages on the client side using HTML5 and CSS3 technologies;
ii) Develop programs that involve the use of the programming language on the 
client side (JavaScript) and relate it to HTML and CSS; and 
iii) Know different JavaScript Technologies (e.g. jQuery) and frameworks used 
in the design phase (e.g. Boostrap).

The second unit comprises: 
i) Describe the historical perspective of the Internet, the family of 
TCP/IP protocols, as well as WWW; 
ii) Know the Client/Server architecture focusing and detailing the operation 
of the HTTP protocol; 
iii) Install web servers and database.

The third unit includes: 
i) Identify existing technologies for server-side web programming; 
ii) Know the paradigm of web programming on the server side (session variables, 
cookies, ajax, etc.); 
iii) Develop programs using a server-side language.

With these three units we cover the main topics and try not to saturate the students.

\subsection{Operating System Simulator as a Project Integrator and Teaching Strategy}
\label{ssec:projectIntegrator}


Thinking about a final project where students can start and expand in terms of design 
and development is not an easy task. However, here we propose a project that has given 
us very good results and that we have been polishing for the last 8 years. 
Our proposal is to develop a Web-based Operating System Simulator. This involves for 
the student to choose an existing Operating System to replicate it in terms of design, 
and to develop some modules so that a simulation is actually reached in terms of 
operability. 

This proposal is divided into two phases: design and development (Units 1 and 3 in 
Section~\ref{ssec:learningUnits}). 
The design phase involves the initial project with Front-end themes such as HTML, 
CSS and JavaScript. 
The development phase involves the final project and is specified when back-end 
technologies have already been taught such as server installation (Unit 2 in 
Section~\ref{ssec:learningUnits}) and server-side programming. 
In both phases, practices are carried out that lead to the synergy of the initial 
project (respectively the final one).




\section{4. Proyecto integrador y su relación con las evidencias generales}
\label{sec:gralIntegratorProject}


%\subsection{Integrator Project}
\subsection{4.1 Projecto Integrador}
\label{ssec:integratorProject}
The whole integrator project is explained to the students in two stages, 
at the beginning and in the middle of the course. In both stages the scopes 
are explained and the prototypes are reviewed on the specified date.

\subsubsection{Initial project: Design of a Web-based Operating System Simulator}
\label{sssec:desginOS}
It is expected that with the specifications of the initial project students 
will achieve their learning in the topics of HTML, CSS, JavaScript and derived 
libraries (Unit 1, Section~\ref{ssec:learningUnits}). In this stage, high design 
content is requested, especially because it seeks to get students to stick as 
closely as possible to the design of the simulated OS, that is where students 
confirm the use of CSS. Some students may download images that are very similar 
to the Operating System they have selected, this is valid because they learn to 
manipulate the use and positioning of images.

Next we explain an example of the indications that are given to the students, then
the knowledge that they acquire and the challenges that they present when 
simulating a Web-based Operating System:
\begin{itemize}
    \item \textbf{Login:} show at the beginning of the OSS an authentication mechanism. 
        Students must apply knowledge in HTML, CSS and Forms. Although not knowing yet
        how to use the GET and POST methods.
    \item \textbf{The desktop:} show the startup desktop. This faces several design 
        challenges considering 
        the different proposals that exist in Operating Systems. Students must apply 
        high knowledge of HTML and CSS.
    \item Develop a menu of \textbf{options} (as all Operating Systems have a main menu of 
        options), that contains at least the following actions:
        \begin{enumerate}
            \item Identity
            \item Calculator; and
            \item Notes
        \end{enumerate}
    \item \textbf{Identity:} it must be placed in some ingenious section of the selected 
        Operating System and must show the personal data of the developer of the page, 
        the logo of the university, the subject, and the name of the teacher.
    \item \textbf{Calculator:} when clicking on this option, a calculator must be displayed
        similar in design to the chosen Operating System, and it must calculate arithmetic 
        operations such as addition, subtraction, multiplication, division and residual (mod). 
        This involves consolidating knowledge in JavaScript, HTML and CSS.
    \item \textbf{Notes:} When clicking on this option, a system of notes (create and delete 
        notes) should appear. As it has not yet seen the issues that involve the server is 
        not requested data persistence. This section allows the student to consolidate their 
        knowledge between JavaScript and their access to the HTML DOM, either by JavaScript 
        primitives or by JQuery.
\end{itemize}

Something important that is asked of the student is that if the operating system he has selected 
does not cover the aforementioned aspects, it must be invented by them or do something additional 
to comply with the specified points.

\subsection{4.2. Prácticas}

\subsubsection{4.2.1. Prácticas del lado del Servidor}
Nowadays it is quite easy to configure web working tools with technologies such as LAMP, XAMP, WAMP 
and MAMP. However, sometimes students lose the focus of these technologies and ignore the use of 
several servers concepts such as the web, database, FTP (s) server tools to access information remotely, 
besides forgetting important aspects such as configuration about the port number, id of the 
session, maximum amount of file upload, etc., which are part of the web paradigm.

For this reason, we propose a first practice that consists in installing a Web Server, and execute 
a program that supports it. In this practice is not requested a particular Web Server, because the main 
idea is that students can explore different technologies. However, something important is to know how 
to install a web server and the path to include files that will be processed by the server and sent
to the browser to be displayed. 

In a second practice, the students must comprehend which part of the script runs on the server and which 
part on the client. To do that, students must deliver a practice very similar with the first one, but now
they must deliver it at at least to different computers: one being the web server and the other one using
the browser.

In the third practice, students must install a database server and understand the APIs required to 
establish communication between the web server, and the database server. It is intended that at a given 
time they can consult specific data from the database.


In a fourth practice, they must mount an FTP server (s) in order to remotely access the data through a client.

It should be taken into account that some companies may have the need to have their own web server. 
That is, not depend on a hosting service outside to host your website or to perform other types of 
tasks.






%% The Appendices part is started with the command \appendix;
%% appendix sections are then done as normal sections
\appendix

\section{Section in Appendix}
\label{appendix-sec1}

Sample text. Sample text. Sample text. Sample text. Sample text. Sample text. 
Sample text. Sample text. Sample text. Sample text. Sample text. Sample text. 
Sample text. 


%% References
%%
%% Following citation commands can be used in the body text:
%% Usage of \cite is as follows:
%%   \cite{key}         ==>>  [#]
%%   \cite[chap. 2]{key} ==>> [#, chap. 2]
%%

%% References with bibTeX database:

\bibliographystyle{elsarticle-num}
% \bibliographystyle{elsarticle-harv}
% \bibliographystyle{elsarticle-num-names}
% \bibliographystyle{model1a-num-names}
% \bibliographystyle{model1b-num-names}
% \bibliographystyle{model1c-num-names}
% \bibliographystyle{model1-num-names}
% \bibliographystyle{model2-names}
% \bibliographystyle{model3a-num-names}
% \bibliographystyle{model3-num-names}
% \bibliographystyle{model4-names}
% \bibliographystyle{model5-names}
% \bibliographystyle{model6-num-names}

\bibliography{sample}


\end{document}

%%
%% End of file `elsarticle-template-num.tex'.
