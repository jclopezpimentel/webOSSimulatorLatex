\section{Conclusions}
\label{sec:Conclusions}

\Caro{Web programming is becoming an indispensable requirement in software development, 
especially for the new generation of entrepreneurs and innovators, who want to be more creators 
than users, and to generate employment with their ideas instead of being employees. 
It is no longer enough to use traditional applications and visit web pages daily,
the real challenge in the coming times is to create more innovative web applications than 
traditional ones.}

In this article we have presented a face to face teaching method for Web Programming course.
It consists in the the development of an Operating System Simulator as an 
integrator project strategy. 
Throughout the document, we have shared our experience and knowledge in the 
application of the methodology and their activities.

The evaluation method has been applied for 9 years with an approximate of 10 different groups, 
in three different universities.
Each year we have made gradual changes and we have also identified better levels of approval, 
although the comparisons are not equivalent, in the results, see section~\ref{sec:results}, 
we have given some recommendations for each of them that we consider very appropriate to improve the 
curricular plan of these university careers, which can be applied to others around the world that
feels similar circumstances.

This research might be used by teachers who are looking for a teaching-learning strategy in 
unique courses in Web Application Development. 
This strategy is appropriate as long as the teacher faces students with prior knowledge of 
structured and object-oriented programming and would be formidable with knowledge of the 
database and Network.

Additionally, we consider that although the purpose of the article is not to verify the 
hypothesis presented in the Introduction, we have validated it in some way since 
generation after generation we have built dynamic materials that allow students to solve 
personalized practices in each course.