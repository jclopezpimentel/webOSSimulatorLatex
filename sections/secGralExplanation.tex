%\section{Web Operating System Simulator as an Evidence in the Teaching-Learning Process}
\section{Operating System Simulator with Web Technology}
\label{sec:GralExplanation}

\subsection{Learning Units}
\label{ssec:learningUnits}
The subject comprises 16 weeks, each week involves 3 face-to-face hours and 
3 extra-class hours. Our restriction is to provide in a single course web 
programming topics of Front-end and Back-end, hence we have conformed the 
following units: 
i) Design, development and programming of web pages on the client's side; 
ii) Internet, WWW and HTTP; and 
iii) Programming of server-side web pages.

The first unit comprises three learning outcomes: 
i) Design web pages on the client side using HTML5 and CSS3 technologies;
ii) Develop programs that involve the use of the programming language on the 
client side (JavaScript) and relate it to HTML and CSS; and 
iii) Know different JavaScript Technologies (e.g. jQuery) and frameworks used 
in the design phase (e.g. Boostrap).

The second unit comprises: 
i) Describe the historical perspective of the Internet, the family of 
TCP/IP protocols, as well as WWW; 
ii) Know the Client/Server architecture focusing and detailing the operation 
of the HTTP protocol; 
iii) Install web servers and database.

The third unit includes: 
i) Identify existing technologies for server-side web programming; 
ii) Know the paradigm of web programming on the server side (session variables, 
cookies, ajax, etc.); 
iii) Develop programs using a server-side language.

With these three units we cover the main topics and try not to saturate the students.

\subsection{Operating System Simulator as a Project Integrator and Teaching Strategy}
\label{ssec:projectIntegrator}


Thinking about a final project where students can start and expand in terms of design 
and development is not an easy task. However, here we propose a project that has given 
us very good results and that we have been polishing for the last 8 years. 
Our proposal is to develop a Web-based Operating System Simulator. This involves for 
the student to choose an existing Operating System to replicate it in terms of design, 
and to develop some modules so that a simulation is actually reached in terms of 
operability. 

This proposal is divided into two phases: design and development (Units 1 and 3 in 
Section~\ref{ssec:learningUnits}). 
The design phase involves the initial project with Front-end themes such as HTML, 
CSS and JavaScript. 
The development phase involves the final project and is specified when back-end 
technologies have already been taught such as server installation (Unit 2 in 
Section~\ref{ssec:learningUnits}) and server-side programming. 
In both phases, practices are carried out that lead to the synergy of the initial 
project (respectively the final one).



