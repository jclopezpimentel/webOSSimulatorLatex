\section{Results}
\label{sec:results}
Our teaching-learning strategy previously presented together with the evaluation  
has been applied for 9 years approximately and with 10 different groups, 
6 at the Polytechnic University of Chiapas (UPChiapas), 4 at the Technological 
Institute of Tuxtla Guti\'errez (ITTG), and 
2 at Panamericana Univerisity, Campus Guadalajara. 
See Table 3 for a statistical summary.

\begin{table}[htb]
    \begin{center}
        \caption{General results}
        \label{table:results}
        \begin{tabular}{|l|c|c|c|c|l|c|}
      \hline
      Group	 &Grade&Num.    &Approved&Not     &University      & \% \\
      Name	 &     &Students&        &Approved&	               &    \\\hline
      A2009	 &6q   &23      &14      &9       &UPChiapas	   &61\%\\	\hline
      A2011  &6q   &34      &23      &11      &UPChiapas	   &68\%\\	\hline
      A2012  &6q   &24      &14      &10      &UPChiapas	   &58\%\\	\hline
      A2013  &6q   &36      &22      &14      &UPChiapas	   &61\%\\	\hline
      A2015  &6q   &40      &28      &12      &UPChiapas	   &70\%\\	\hline
      A2016  &6q   &30      &21      &9       &UPChiapas	   &70\%\\	\hline
      A2017	 &9s   &18      &14      &4       &TecTuxtla	   &78\%\\	\hline
      B2017	 &9s   &27      &26      &1       &TecTuxtla	   &96\%\\	\hline
      C2017	 &9s   &30      &22      &8       &TecTuxtla	   &73\%\\	\hline
      A2018	 &9s   &30      &28      &2       &TecTuxtla	   &93\%\\	\hline
      DAW6344&3s   &14      &14      &0       &UPanamericana   &100\%\\	\hline
      DAW6345&6s   &16      &14      &2       &UPanamericana   &88\% \\	\hline
        \end{tabular}
    \end{center}
\end{table}

We have made gradual changes in the way we have imparted this strategy with each group. 
These include adjustments in topics content due to the emergence of new technologies 
and the differences in the duration of each applied course 
(see column Grade in Table~\ref{table:results}, \textbf{q}uarter or \textbf{s}emester).
Hence, the difference (improvement) in the approval percentage. 
For example, the last groups have been applied in the TecTuxtla and UPanamericana, 
where we can observe the best approval index. Although we must clarify that the 
courses given in these last Universities have longer duration than in the UPChiapas.
In addition; in the semester that this subject is imparted in the TecTuxtla the 
students have already taught Database topics, whereas in the UPChiapas nor DAW6345 of 
Panamericana University not yet. Another clarification is that in the courses of 
UPanamericana, students have not taken topics of network, thence, the students 
were a bit hampered with client-server topics.

The analysis on what we have arrived is that, in the case of Polytechnic Universities, 
it is important to divide the course in two as suggested by Ye Diana Wang et.al. in~\cite{9Wang2009} 
and Xusheng Wang~\cite{8Wang2014}. 
In the case of TecTuxtla that already have two courses focused on the development of web 
applications it is recommended to pass the Web Programming course in a previous semester and 
in the Advanced Topics in Web Programming Technologies Course in a second one but in a 
following semester (that is, the last one in this case). UPanamericana is suggested to include 
Network and Database subjects before of the Web subject and to split the course
in two as suggested above. 

Making these changes would be possible to include methodologies as suggested by Harrigier and 
Woods~\cite{3Harriger2001}, where a teaching method based on the development of websites for local 
businesses is proposed, together with a web-based software development methodology. 
Another work that suggests something similar is that of Margaret et.al.,~\cite{6Margaret2016}, 
where the development of a portal web with teams of two or three students is proposed.