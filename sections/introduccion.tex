\section{Introduction}
\label{sec:intro}

Currently the Web facilitates our life in several aspects. Applications such as 
email, chat, social networks, among others, allow people to be more connected 
and interrelated than in previous times; Internet banking, electronic commerce 
are examples of applications that save us time and effort. In terms of getting 
information we have also been favored, which a while ago it took us a long time 
to know, nowadays with the search engines allow us to have the information in a 
short time.

In education, the Web provides a wide range of resources and services focused on 
the teaching-learning process, but it also causes several challenges. For example, 
validate that the information is reliable. Many solutions of activities or class 
projects can be found on the Web, this being a real challenge for teachers, since 
the extra-class activities left by teachers could be being by the students, then 
the teacher would not have the certainty that the students are achieving the goal.

So, this era with the Internet demands and challenges teachers to reform their 
traditional courses and even more if the subject is related to programming or 
development of Web applications. The Web is reinvented every day, what was once 
considered technically impossible to do, nowadays it is easier, since with the 
passage of time the technologies used for their development have been innovated. 
Thus, teaching Web applications is a changing topic that can not remain static.

Some researchers emphasize much about the main challenges in web programming 
\cite{9Wang2009,7XinogalosK2012,4Liu2011}. Other authors have proposed different 
strategies to deal with it; some propose technologies that should be taught in a 
single course, \cite{4Liu2011,10Rosenbloom2017}, but with prior knowledge. Others 
propose dividing it in two \cite{9Wang2009}. In our case, we have considered some 
of such proposals, which will explained below, but focused on a single course.

We are inspired by the following hypothesis: with the customized development of a 
web-based operating system simulator, we can find the formula to generate dynamic 
solutions and thus achieve the necessary skills that students must acquire in Web 
Programming in such demanding times.

Our proposal is suitable as long as the students are trained in topics such as 
structured programming, object-oriented programming, data structure, and database 
notions. If the students already have some knowledge in web technologies it is 
desirable but it is not necessary.

The general objective of this article is to provide facilitators, and even students, 
a teaching-learning methodology for subjects related to Web Programming through 
strategies that combine the knowledge and experience of the teacher with an 
integrating project (simulator of an operating system) that students generate in a 
personalized way from the beginning to the end of the subject. In addition, students 
must submit activities, which we call evidences that are attached to the course 
syllabus and the integrating project, that students must complete to pass. All this 
will eventually lead to better skills in the students.

Although this proposal has been applied in different scenarios, the proposal 
presented here has been adjusted to 48 contact hours and 48 non-contact hours. 
To combat the restriction of time considering the amount of web technologies that 
currently exist and give students the main motivation to learn the State of the Art 
in this topic, we have carefully selected a set of important and representative topics 
that eventually lead to a set of tools currently used by developers.

%Sample text. Sample text. Sample text. Sample text. Sample text. Sample text. 
%Sample text. Sample text. Sample text. Sample text. Sample text. Sample text. 
%Sample text. Sample text. Citation of Einstein paper~\cite{Einstein}.

% \subsection{Sample subsection}
% \label{subsec1}

% Sample text. Sample text. Sample text. Sample text. Sample text. Sample text. 
% Sample text. Sample text. Sample text. Sample text. Sample text. Sample text. 
% Sample text. 
