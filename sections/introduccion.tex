\section{Introduction}
\label{sec:intro}

\Caro{The latest advances in the Internet have revolutionized the way in which business information systems are conceived, and the Web has become the framework for publishing and exploiting all types of multimedia contents and resources, which constitute the fundamental channel of communication between the company and the consumer} \cite{cabada2018affective}.

The web facilitates our life in several aspects. Applications such as email, chat, social networks, among others, allow people to be more connected and interrelated than in previous times. Internet banking, electronic commerce are examples of applications that save us time and effort. In terms of getting 
information we have also been favored, which a while ago it took us a long time to know, nowadays with the search engines allow us to have the information in a short time. 

In education, the Web provides a wide range of resources and services focused on the teaching-learning process, but it also causes several challenges. For example, %validate that the information is reliable. 
teachers must design new strategies or ways of homeworks.
Many solutions of activities or class projects can be found on the web, this being a real challenge for teachers, since the extra-class activities could be being copied by students, then the teacher would not have the certainty that they are achieving the goal. Amber Walraven et.al., \cite{Walraven2009} 
carry out a study about how students solve information problems and what kind of criteria they use 
when evaluating results, sources and information while searching information in 
the Internet, they concludes that students spent most of their time on searching and 
scanning and only a small amount of time on processing and organizing information.

On the other hand, \Caro{the positioning of web programming as a priority issue for children and adults, has given rise to different online schools that seek to encourage learning and improve the training of those who want to learn or work on the subject.}

So, this era with the Internet demands and challenges to teachers to reform their 
traditional courses and even more if the subject is related to programming or 
development of web applications. The web is reinvented every day, what was once 
considered technically impossible to do, nowadays it is easier, since with the 
passage of time the technologies used for their development have been innovated. 
Thus, teaching web applications is a changing topic that can not remain static.

Some researchers emphasize much about the main challenges in teaching web programming 
\cite{9Wang2009,7XinogalosK2012,4Liu2011,Douce2018}. 
Some of them have proposed different strategies to deal with it; for example, some 
authors propose technologies that should be taught in a single course, 
\cite{4Liu2011,10Rosenbloom2017}, but with prior knowledge. 
Others propose to divide a web course in two, \cite{9Wang2009}, front and back-end 
respectively. 
We have considered some of such proposals, which will explained below, but focused on 
a single course.

Although our research is qualitative, we are inspired by the following hypothesis: 
with the customized development of a web-based operating system simulator, we can 
find the formula to generate dynamic solutions and thus achieve the necessary skills 
that students must acquire in web programming in current demanding times.

Our proposal is suitable as long as the students are previously trained in topics 
such as structured programming, object-oriented programming, data structure, and database 
notions. If the students already have some knowledge in web technologies it would be 
desirable, although it is not necessary.

The aim of this paper is to provide to facilitators, and even students, 
a teaching-learning methodology for subjects related with web programming through 
strategies that combine the knowledge and experience of the teacher with an 
integrative project, an Operating System (OS) simulator, that students generate in a 
personalized way from the beginning to the end of the course. 
Students must submit activities, which we call evidences, that are tied to the  
syllabus and also to the integrative project. 
All of these will eventually lead to better skills in students.

Although this proposal has been applied in different scenarios, the proposal 
presented here has been adjusted to 48 contact hours and 48 non-contact hours. 
To combat the restriction of time considering the amount of web technologies that 
currently exist and give students the main motivation to learn the State of the Art 
in this topic, we have carefully selected a set of important and representative topics 
that eventually lead to a set of tools currently used by developers.

The rest of this document is organized as follows: 
First, we give a brief summary of some related works.
Later (Section~\ref{sec:GralExplanation}), we explain the methodological proposal 
    together with the learning units and the evidences that the students must meet.
Afterwards, we explain our method emphasizing the different activities together with 
    the web technology that students are expected to learn, 
    Section~\ref{sec:gralIntegratorProject}. 
After having applied this method for 9 years, we discuss some qualitative results in 
    Section~\ref{sec:results}.
Finally we give some conclusions about this work.

%\subsection{\Caro{Motivation}}

%\Caro{The use of educational technology plays an important role because it is a repository that allows access to materials immediately and permanently, in addition to providing other asynchronous communication tools that allow students to communicate with each other and / or with the teacher in a timely manner; giving them greater control over the resources and spaces available and collaborative tools that make it possible to share resources, learn and understand the bases of programming in an easy and entertaining way.}
