\section{Related Works}
There are several works explaining the difficulty of teaching web programming and others
regarding with proposing different strategies of teaching it.

\subsection{The Challenge of Teaching Web Programming}
Wang and Zahadat,~\cite{9Wang2009}, conduct a study where they emphasize that the explosive growth 
of Web 2.0 Technologies present a significant challenge for teachers focused on teaching web development. 

Wang et.al. ,\cite{wang2009teaching}, present an IT educator's perspective and
describe some challenges and problems of teaching Web development. Some of the approaches that they
present are: Web development to user interaction, client-side and server-side Web development, 
real-world applications, and constructivist teaching methods.

Stelios Xinogalos and Theodores H.~\cite{7XinogalosK2012}, in a very similar sense, make an  important study about the main challenges in the teaching of web programming. 
Yi Liu and Gita Phelps,~\cite{4Liu2011} present a similar study in terms of challenges and also 
include important tools that should be used in the teaching of web programming. 
They focus on a junior level similar to our case, but they are more busy teaching the 
foundations of web programming and sharing their experiences more than the technologies 
themselves.

All previous authors agree that teaching web programming is a challenging task, and for example, 
Chris Douce,~\cite{Douce2018}, recently carried out a research about understanding the tutor's 
perspective in the teaching of web technologies, interviewing 12 teachers and agreeing that the 
issues of programming, solving problem skills, and also to include several technologies in a course, 
is a challenging topic not only for students but also for teachers.

\subsection{Teaching Strategies in Web Programming}
Wang and Zahadat,~\cite{9Wang2009}, have developed a method consisting of 4 elements: 
a) Concentrate on AJAX; 
b) Divide the course in two: one focused on the client and one on the server; 
c) Assign projects that integrate topics with real world applications; and 
d) Use constructive teaching methods. 
Although what they propose is very interesting and useful, some curricular plan of studies
include a single subject related with web programming, as is our case. 

Rosenbloom and Yueli,~\cite{10Rosenbloom2017}, propose a twelve-week course that 
involves starting to see Model View Controller (MVC) frameworks and includes web services topics 
as rest-ful, but specifically specifies that students should have prior or basic knowledge.

Francesco Maiorana,~\cite{Maiorana14}, discusses a case study based on the administration of a 
logging system, covering with this, main topics of web programming, databases (transactions, 
stored procedures) and security aspects. 
We take into account this methodology from the beginning of our practices until the end, however, 
we do not take into account security aspects in depth for the time being.

Unlike Wang and Zahadat,~\cite{9Wang2009}, (mentioned above) our main challenge is to adjust all 
subjects in a single course that contains 48 contact hours and 48 extra-class hours (96 in total). 
We know that it is not easy for students with little experience in web programming to absorb so much 
information in a short period of time. 
However, our objective will be to teach main and current tools so that in the future they can quickly 
migrate to other web technologies.

\subsection{Web Programming in E-commerce and Gamming}

%\subsection{\Caro{Learning Sources}}
%\label{ssec:learningSources}

\Caro{Information development and communication technologies has allowed learning to take place not only through educational institutions, but through non-formal environments related to the Internet, which differ in their respective approaches and uses as tools} \cite{gerardo2017fundamentos}.

%\subsection{\Caro{Serious games}}
%\label{ssec:seriousGames}

\Caro{A serious game is one that focuses on learning, instead of entertainment. According to Chen and Micheal, in} \cite{michael2005serious}\Caro{, these games have more than history, art and software, they imply pedagogy. It is an activity that educates through the acquisition of skills. Serious games can provide immersive learning opportunities and some seem crucial to the competencies required for modern citizens and business professionals and industries in the current information age.}


\Caro{
Web programming allows the creation of dynamic sites on the Internet. This is achieved by generating the contents of the site through a database through Web programming languages}, \cite{hebeler2011semantic}. 
 \Caro{This concept revolutionizes all areas, since it is not only necessary to know how different websites or applications are used, but also to learn how these have been created. For example, online training schools have been greatly encouraged. This learning also includes web design, graphic design, networks and security.}

\Caro{Students need to gather and organize their efforts and improve the learning tools to show their competence. For example, This era is experiencing different forms of Web interpretation and study, providing instructive details of our pedagogical strategies and approaches to the design and delivery of web development. }

\Caro{There is a wide variety of web 2.0 tools for the development of e-activities. Some of its advantages are: ease of use, the possibility of interacting in real time, the opportunity to create dynamic learning communities and to create immersive experiences through 3D scenarios.
}





