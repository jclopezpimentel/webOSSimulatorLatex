\section{4. Proyecto integrador y su relación con las evidencias generales}
\label{sec:gralIntegratorProject}


%\subsection{Integrator Project}
\subsection{4.1 Projecto Integrador}
\label{ssec:integratorProject}
The whole integrator project is explained to the students in two stages, 
at the beginning and in the middle of the course. In both stages the scopes 
are explained and the prototypes are reviewed on the specified date.

\subsubsection{Initial project: Design of a Web-based Operating System Simulator}
\label{sssec:desginOS}
It is expected that with the specifications of the initial project students 
will achieve their learning in the topics of HTML, CSS, JavaScript and derived 
libraries (Unit 1, Section~\ref{ssec:learningUnits}). In this stage, high design 
content is requested, especially because it seeks to get students to stick as 
closely as possible to the design of the simulated OS, that is where students 
confirm the use of CSS. Some students may download images that are very similar 
to the Operating System they have selected, this is valid because they learn to 
manipulate the use and positioning of images.

Next we explain an example of the indications that are given to the students, then
the knowledge that they acquire and the challenges that they present when 
simulating a Web-based Operating System:
\begin{itemize}
    \item \textbf{Login:} show at the beginning of the OSS an authentication mechanism. 
        Students must apply knowledge in HTML, CSS and Forms. Although not knowing yet
        how to use the GET and POST methods.
    \item \textbf{The desktop:} show the startup desktop. This faces several design 
        challenges considering 
        the different proposals that exist in Operating Systems. Students must apply 
        high knowledge of HTML and CSS.
    \item Develop a menu of \textbf{options} (as all Operating Systems have a main menu of 
        options), that contains at least the following actions:
        \begin{enumerate}
            \item Identity
            \item Calculator; and
            \item Notes
        \end{enumerate}
    \item \textbf{Identity:} it must be placed in some ingenious section of the selected 
        Operating System and must show the personal data of the developer of the page, 
        the logo of the university, the subject, and the name of the teacher.
    \item \textbf{Calculator:} when clicking on this option, a calculator must be displayed
        similar in design to the chosen Operating System, and it must calculate arithmetic 
        operations such as addition, subtraction, multiplication, division and residual (mod). 
        This involves consolidating knowledge in JavaScript, HTML and CSS.
    \item \textbf{Notes:} When clicking on this option, a system of notes (create and delete 
        notes) should appear. As it has not yet seen the issues that involve the server is 
        not requested data persistence. This section allows the student to consolidate their 
        knowledge between JavaScript and their access to the HTML DOM, either by JavaScript 
        primitives or by JQuery.
\end{itemize}

Something important that is asked of the student is that if the operating system he has selected 
does not cover the aforementioned aspects, it must be invented by them or do something additional 
to comply with the specified points.

\subsection{4.2. Prácticas}

\subsubsection{Practices in the Server Side}
Nowadays it is quite easy to configure web working tools with technologies such as LAMP, XAMP, WAMP 
and MAMP. However, sometimes students lose the focus of these technologies and ignore the use of 
servers concepts such as the web server, DBMS, FTP server, 
besides forgetting important aspects such as configuration about the ip, port number, id of the 
session, maximum amount of file upload, etc., which are part of the web paradigm.

For this reason, we propose a first practice that consists in installing a Web Server. 
%In this practice 
%is not requested a particular Web Server, because the main idea is that students can explore different 
%technologies. However, something important is to know how to install a web server and how to upload the 
%files that will be processed by the server. 
The student must deliver a very simple practice with code executing in the server side and another 
executing in the client side.\footnote{See an example of this practice at 
    \url{\pathpractices server01.pdf}} 

In a second practice, the students must comprehend which part of the script runs on the server and which 
part on the client. To do that, students must deliver a practice very similar with the first one, but now
they must deliver it at least to different computers: one being the web server and the other one using
the browser.\footnote{See an example of this practice at 
    \url{\pathpractices server02.pdf}}

In a third practice, students must install a database server and understand the APIs required to 
establish communication between the web server, and the database server. It is intended that at a given 
time they can consult specific data from the database.\footnote{See an example of this practice at 
    \url{\pathpractices server03.pdf}}

It is important that students have some bases on networks and installation of servers. We have identified that
some students have difficulties to understand these topics when terms about ip, ports, client side and server side
is explained within the classroom.
%In a fourth practice, they must mount an FTP server (s) in order to remotely access the data through a client.

%It should be taken into account that some companies may have the need to have their own web server. 
%That is, not depend on a hosting service outside to host your website or to perform other types of 
%tasks.

